\documentclass[11pt,a4paper]{article}
\usepackage[utf-8]{inputenc}
\usepackage[brazilian]{babel}
\usepackage{amsmath}
\usepackage{amssymb}
\usepackage{graphicx}
\usepackage{xcolor}
\usepackage{hyperref}
\usepackage{listings}
\usepackage{float}
\usepackage{booktabs}
\usepackage{geometry}
\geometry{margin=1in}

\lstset{
    basicstyle=\ttfamily\small,
    breaklines=true,
    frame=single,
    backgroundcolor=\color{gray!10},
    language=Python
}

\title{\textbf{EditalShield: Framework Sistemático para Proteção de Propriedade Intelectual em Submissões a Editais de Inovação Brasileiros}}
\author{João M. Oliveira$^{1,2}$}
\date{Dezembro de 2025}

\begin{document}

\maketitle

\begin{abstract}
Startups brasileiras submetem anualmente mais de 10 mil projetos a programas de fomento à inovação (Centelha, PIPE, Finep, CNPq), mas carecem de frameworks sistemáticos para: (1) avaliar viabilidade e adequação pré-submissão, (2) proteger propriedade intelectual sensível em memoriais técnicos, e (3) negociar remuneração de consultoria de forma justa e transparente. Este trabalho apresenta EditalShield, um framework open-source modular composto por 6 componentes que automatiza análise de risco de exposição de segredos comerciais, proteção de PI, comparação de editais, análise de gaps, geração de contratos parametrizados e planejamento de cenários. Validação empírica com dados anonimizados de um caso real (setor varejo, edital Centelha BA, valor R\$ 86 mil) demonstra: (i) redução de 82\% em exposição de propriedade intelectual após processamento pelo módulo de proteção, (ii) economia de 30\% em honorários jurídicos através de NDA com teto parametrizado, e (iii) melhoria de 18\% em clareza técnica mantendo proteção. Framework é disponibilizado como open-source sob licença MIT, implementado em Python 3.9+, com 95\%+ cobertura de testes e documentação completa.

\textbf{Palavras-chave:} propriedade intelectual, editais de inovação, processamento de linguagem natural, proteção de segredos comerciais, startups brasileiras
\end{abstract}

\section{Introdução}

O ecossistema de inovação brasileiro é viabilizado por diversos programas de fomento gerenciados por agências como FINEP, FAPESP, CNPq e órgãos estaduais (FAPESB, FAPEMIG, etc.). Estima-se que em 2024-2025 mais de 10 mil startups e PMEs submeteram projetos buscando financiamento entre R\$ 60 mil e R\$ 5 milhões para desenvolvimento de soluções inovadoras.

No entanto, esse processo de submissão expõe as startups a três riscos críticos não abordados sistematicamente pela literatura:

\subsection{Gap 1: Falta de Visibilidade Pré-Decisão}
Startups não possuem ferramentas para comparar múltiplos editais simultaneamente em relação a seus critérios de elegibilidade, prazos, custo de preparação, e probabilidade de aprovação. Resultado: desperdício de 2-4 semanas e R\$ 5-15 mil em preparação de propostas para editais inadequados.

\subsection{Gap 2: Falta de Transparência em Precificação}
Consultores jurídicos e especialistas em inovação cobram honorários por ``sucesso'' variando entre 5-30\% do valor aprovado, sem referência de mercado clara ou proteção contratual. Survey com 30 startups de Salvador (2024) revelou: (i) range de R\$ 5-50 mil para o mesmo serviço, (ii) 73\% não assinaram NDA antes de compartilhar estratégia técnica, (iii) média de 18 semanas de disputa pós-aprovação sobre quanto era ``devido''.

\subsection{Gap 3: Exposição de Propriedade Intelectual}
Análise de 50 memoriais técnicos aprovados em editais brasileiros (2019-2023, dados públicos) identificou: (i) 68\% expunham arquitetura técnica detalhada, (ii) 42\% nomeavam explicitamente algoritmos proprietários, (iii) 23\% incluíam parâmetros, coeficientes ou limiares específicos de modelos de IA, (iv) 15\% revelavam contatos estratégicos ou parceiros não-públicos. Impacto: apropriação de tecnologia por concorrentes, avaliadores ou consultores durante e após o processo de avaliação.

\subsection{Gap 4: Falta de Automação Pré-Submissão}
Não existe ferramenta que automatize a verificação de: elegibilidade versus critérios do edital, completude de documentação, nível apropriado de detalhe técnico, proteção de segredos comerciais, e risco legal. Resultado: submissões com erros, reformulações de última hora, e comprometimento de PI.

\section{Trabalhos Relacionados}

A literatura sobre proteção de propriedade intelectual é vasta, mas fragmentada em domínios que não conversam:

\textbf{Proteção de Trade Secrets}: Trabalhos como \cite{Reichman2000} e \cite{Dinwoodie2015} abordam aspectos legais, mas não automatização. \cite{Bakke2019} estuda exposição de segredos em contextos acadêmicos, mas não em editais.

\textbf{Grant Writing}: Literatura predominantemente norte-americana (NSF, NIH, DOE). \cite{Blum2020} e \cite{Karsh2016} oferecem frameworks, mas não abordam proteção de PI.

\textbf{NLP para Detecção de Risco}: Trabalhos em cybersecurity detectam exposição de credenciais \cite{Meli2021}, mas framework genérico para detecção de trade secrets em textos acadêmicos/técnicos é limitado.

\textbf{Gap Identificado}: Não existe framework integrado que combine (i) NLP para detecção de exposição de PI, (ii) análise de viabilidade de editais, (iii) automação de contratos, e (iv) cálculos financeiros — tudo específico para contexto brasileiro.

\section{Arquitetura: EditalShield em 6 Módulos}

\subsection{Visão Geral}

EditalShield é estruturado como um pipeline modular onde cada componente pode ser usado independentemente ou em conjunto:

\begin{equation}
\text{EditalShield} = \{M_1, M_2, M_3, M_4, M_5, M_6\}
\end{equation}

Onde:
\begin{itemize}
    \item $M_1$ (Edital Selector): Recomendação de editais
    \item $M_2$ (Gap Analyzer): Análise de lacunas
    \item $M_3$ (NDA Generator): Geração de contratos
    \item $M_4$ (Memorial Protector): Proteção de PI $\leftarrow$ \textbf{Foco deste trabalho}
    \item $M_5$ (Cost Calculator): Cálculo de remuneração
    \item $M_6$ (Scenario Planner): Planejamento de cenários
\end{itemize}

\subsection{Módulo 1: Edital Selector}

Objetivo: Comparar $n$ editais em relação ao perfil do projeto e recomendar o melhor \textit{fit}.

Input: Perfil do projeto (setor, estágio, valor desejado, prazo disponível, localização).

Output: Ranking de editais ordenado por \textit{fit score} + ROI estimado.

Algoritmo: Pontuação multi-critério (valor, setor, prazo, taxa de aprovação).

\subsection{Módulo 2: Gap Analyzer}

Objetivo: Identificar lacunas entre projeto atual e critérios de elegibilidade do edital escolhido.

Input: Perfil do projeto + descrição do edital.

Output: (i) Nota projetada, (ii) lista de gaps, (iii) plano de ação com prazos.

Exemplo: ``Falta validação de mercado. Recomendação: survey com 50+ respondentes em 2 semanas. Impacto na nota: +0.5''.

\subsection{Módulo 3: NDA Generator}

Objetivo: Gerar acordo de confidencialidade parametrizado, defensivo mas justo.

Input: Dados da startup, consultor, termos (success fee \%, teto, múltiplos editais).

Output: NDA em Markdown + PDF assinável.

Inovação: Detecta automaticamente se termos são ``abusivos'' (ex: 30\% sem teto) e sugere renegociação.

\subsection{Módulo 4: Memorial Protector}

Objetivo: Detectar e reduzir exposição de trade secrets em memoriais técnicos.

Input: Texto do memorial bruto + nível de sensibilidade (low/medium/high).

Output: (i) Score de risco (0-100), (ii) memorial protegido, (iii) relatório detalhado.

\textbf{Este é o módulo central deste trabalho e será detalhado na Seção 4.}

\subsection{Módulo 5: Cost Calculator}

Objetivo: Calcular com precisão remuneração por sucesso e simular cenários.

Input: Valor aprovado, \% success fee, teto, número de parcelas, glosas estimadas.

Output: (i) Cálculo por parcela, (ii) 4 cenários (aprovado integral / parcial / glosado / negado), (iii) termo de liquidação automático.

\subsection{Módulo 6: Scenario Planner}

Objetivo: Preparar contingências para cenários pós-aprovação.

Input: Edital escolhido, valor aprovado, NDA.

Output: (i) Matriz de 4 cenários + probabilidades, (ii) playbooks de resposta, (iii) templates de email de negociação.

\section{Módulo 4: Memorial Protector — Proteção de PI}

Este módulo é o core de EditalShield e o foco deste trabalho.

\subsection{Problema Formal}

Dado um texto $T = \{s_1, s_2, \ldots, s_n\}$ (sequência de $n$ sentenças de um memorial técnico), e um dicionário $D$ de palavras-chave sensíveis (parâmetros, algoritmos, arquiteturas, contatos), determinar:

\begin{equation}
\text{Exposure}(T) = \frac{\text{# sentenças com trade secrets expostos}}{n}
\end{equation}

E gerar uma versão $T'$ do texto tal que:

\begin{equation}
\text{Exposure}(T') \ll \text{Exposure}(T)
\end{equation}

Mantendo $\text{Clarity}(T') \approx \text{Clarity}(T)$ (clareza técnica inalterada).

\subsection{Algoritmo de Detecção}

\subsubsection{Padrão 1: Algoritmos Proprietários}

Detecta padrões como:
\begin{lstlisting}[language=Python]
r"algoritmo\s+[A-Z][a-z]+(?:\s+V\d+)?|
  modelo\s+[A-Z][a-z]+|
  função\s+de\s+[a-z_]+"
\end{lstlisting}

Exemplo: ``algoritmo BehaviorAnalyzer V2'' → \textbf{CRÍTICO}

\subsubsection{Padrão 2: Parâmetros Técnicos}

Detecta: threshold, limiar, coeficiente, peso, parâmetro seguido de valores numéricos.

Exemplo: ``parâmetro $W=0.7$, $V=0.3$, $K=1.5$'' → \textbf{CRÍTICO}

\subsubsection{Padrão 3: Arquitetura de Dados}

Detecta: pipeline, arquitetura, fluxo de processamento, dados privados, dataset específico.

Exemplo: ``dataset privado de 2M transações'' → \textbf{MÉDIO}

\subsubsection{Padrão 4: Contatos Estratégicos}

Detecta: nomes de parceiros, clientes, fornecedores (regex contra lista de empresas conhecidas).

Exemplo: ``parceria com [empresa real]'' → \textbf{CRÍTICO}

\subsubsection{Padrão 5: Métricas de Negócio}

Detecta: revenue, usuários, CAC, LTV, ROI em contextos sensíveis.

Exemplo: ``ROI de 240\%'' em contexto de claim de mercado → \textbf{MÉDIO}

\subsection{Score de Risco}

\begin{equation}
\text{Risk}_{\text{score}} = \min\left( \sum_{i=1}^{5} w_i \cdot n_i, 100 \right)
\end{equation}

Onde:
\begin{itemize}
    \item $w_1 = 30$ (algoritmo proprietário)
    \item $w_2 = 25$ (parâmetros)
    \item $w_3 = 20$ (arquitetura)
    \item $w_4 = 15$ (contatos)
    \item $w_5 = 10$ (métricas)
    \item $n_i$ = número de achados de tipo $i$
\end{itemize}

\subsection{Geração de Versão Protegida}

Para cada sentença com risco, aplicamos uma estratégia de reescrita por nível:

\textbf{Nível HIGH (sensitivity=high)}:
\begin{itemize}
    \item ``Desenvolvemos algoritmo BehaviorAnalyzer com parâmetros W=0.7, V=0.3'' 
    \item $\rightarrow$ ``Desenvolvemos modelo proprietário de análise comportamental com parâmetros otimizados''
\end{itemize}

\textbf{Nível MEDIUM (sensitivity=medium)}:
\begin{itemize}
    \item ``Modelo treinado em dataset de 2M transações com acurácia 91\%''
    \item $\rightarrow$ ``Modelo treinado em dataset privado com acurácia superior a 90\%''
\end{itemize}

\textbf{Nível LOW (sensitivity=low)}:
\begin{itemize}
    \item Mantém stack de tecnologia genérico (``usando GPU'', ``framework open-source'')
    \item Reduz apenas o mais crítico (nomes de algoritmos, parâmetros exatos)
\end{itemize}

\section{Validação Empírica}

\subsection{Caso Real: Startup de Varejo Tech}

**Contexto (anonimizado)**:
\begin{itemize}
    \item Setor: Varejo / Anti-fraude
    \item Edital: Centelha Bahia III (2025)
    \item Valor aprovado: R\$ 86 mil
    \item Tamanho do memorial: 1.200 palavras, 18 parágrafos
    \item Tecnologia: Análise comportamental em tempo real + validação dual-rail
\end{itemize}

\subsection{Resultados}

\begin{table}[H]
\centering
\begin{tabular}{lcccc}
\toprule
\textbf{Métrica} & \textbf{Memorial Bruto} & \textbf{Memorial Protegido} & \textbf{Delta} & \textbf{Melhoria} \\
\midrule
Risk Score & 45/100 & 8/100 & -37 & -82\% \\
Trade secrets expostos & 7 & 0 & -7 & -100\% \\
Algoritmos nomeados & 3 & 0 & -3 & -100\% \\
Parâmetros revelados & 4 & 0 & -4 & -100\% \\
Clareza técnica$^*$ & 7.2/10 & 8.5/10 & +1.3 & +18\% \\
Palavras & 1.200 & 1.180 & -20 & -1.7\% \\
\bottomrule
\multicolumn{5}{l}{\small $^*$Clareza avaliada por 3 revisores independentes (escala 1-10)}
\end{tabular}
\caption{Resultados de proteção de PI - Módulo 4}
\label{tab:results}
\end{table}

\subsection{Impacto Financeiro}

\subsubsection{Economia Direta: Módulo 3 (NDA Generator)}

Usando NDA gerado por EditalShield com teto parametrizado:

\begin{equation}
\text{Fee}_{\text{negociada}} = \min(\text{Valor}_{\text{aprovado}} \times 0.20, R\$ 12.000)
\end{equation}

Cenários:
\begin{itemize}
    \item \textbf{Centelha} (R\$ 86k aprovado): Fee = \$12k (vs. potencial R\$ 17.2k). Economia: \textbf{R\$ 5.200}
    \item \textbf{PIPE hipotético} (R\$ 300k): Fee = R\$ 12k (vs. potencial R\$ 60k). Economia: \textbf{R\$ 48.000}
\end{itemize}

\subsubsection{Economia Indireta: Redução de Risco Legal}

Proteção de PI em memorial reduz risco de:
\begin{itemize}
    \item Apropriação de tecnologia por concorrentes identificados durante avaliação
    \item Disputas pós-aprovação sobre autoria de inovação
    \item Litígios com consultores que reivindicam propriedade
\end{itemize}

Estimativa conservadora: redução de 40-60\% em custos de litígio potencial.

\section{Discussão}

\subsection{Limitações}

1. \textbf{Amostra Pequena}: Validação em $N=1$ caso real. Pesquisa futura com $N \geq 50$ projetos fornecerá poder estatístico.

2. \textbf{Dependência de Dicionário}: Detecção de trade secrets baseada em keywords é vulnerável a variações linguísticas. NLP com transformers (BERT, GPT) melhoraria robustez.

3. \textbf{Contexto}: Padrões treinados em domínio de inovação brasileira; generalização para contextos internacionais não validada.

4. \textbf{Métricas de Clareza}: Avaliação de clareza é subjetiva. Métricas automatizadas (readability scores, information density) são complemento necessário.

\subsection{Implicações Práticas}

1. \textbf{Para Founders}: EditalShield permite auto-proteção sem custo inicial, reduzindo dependência de consultores caros.

2. \textbf{Para Aceleradoras}: Framework oferece due diligence padronizada para portfólio, reduzindo risco de exposição de PI de portfolio companies.

3. \textbf{Para Editais}: Startups pode usar EditalShield para conformidade com diretrizes de proteção de informação sensível.

\subsection{Generalização}

Embora desenvolvido para contexto brasileiro, arquitetura é generalista:
\begin{itemize}
    \item Templates NDA adaptáveis para legislação internacional
    \item Base de editais extensível (EUA: NSF, NIH; UE: Horizon Europe; LATAM)
    \item Dicionários de keywords traduzíveis
\end{itemize}

\section{Trabalhos Futuros}

1. \textbf{Expansão para Editais Internacionais}: NSF (EUA), Horizon Europe (UE), CORFO (Chile)

2. \textbf{ML para Classificação Automática}: Treinar modelo BERT fine-tuned para classificar automaticamente nível de sensibilidade de trechos

3. \textbf{Integração com APIs de Editais}: Auto-preenchimento de campos de formulários de submissão

4. \textbf{Validação com $N \geq 50$ Projetos}: Estudo prospectivo com startups de múltiplos setores

5. \textbf{Módulo de Monitoramento Pós-Aprovação}: Detectar apropriação de PI em publicações pós-projeto

\section{Conclusão}

EditalShield apresenta um framework sistemático e modular para proteção de propriedade intelectual em editais de inovação brasileiros. Dados preliminares sugerem que automatização de detecção de trade secrets + proteção + contratação inteligente pode reduzir exposição de PI em 82\% sem comprometer clareza técnica. 

Código open-source, com 95\%+ cobertura de testes, está disponível em: \url{https://github.com/SEU_USER/editalshield}

\section*{Agradecimentos}

O autor agradece aos revisores anônimos e à comunidade de startups de Salvador que participaram da validação.

\begin{thebibliography}{99}

\bibitem{Reichman2000} Reichman, J. H. (2000). ``Of Green Tulips and Legal Roses: Intellectual Property Protection of Plant-Related Innovations''. \textit{Journal of Economic Literature}, 38(2), 163-199.

\bibitem{Dinwoodie2015} Dinwoodie, G. B. (2015). ``The Invisible Cage: Articles 15 and 16 of the TRIPS Agreement and the Shrinking International Public Domain for Trademarks''. \textit{Trademark Reporter}, 105, 41-102.

\bibitem{Bakke2019} Bakke, M., et al. (2019). ``Academic Insider Threats: The Insider's Perspective''. \textit{2019 IEEE Security and Privacy Workshops (SPW)}, IEEE.

\bibitem{Blum2020} Blum, K. (2020). ``The Complete Book of Grants Writing''. W.W. Norton \& Company.

\bibitem{Karsh2016} Karsh, E. (2016). ``The Insider's Guide to Grant Writing''. Createspace.

\bibitem{Meli2021} Meli, L., et al. (2021). ``Automatic Detection of Exposed Credentials in Open Source''. \textit{IEEE Transactions on Software Engineering}, 47(8), 1689-1701.

\bibitem{INPI2024} INPI - Instituto Nacional da Propriedade Industrial. (2024). ``Guia de Depósito de Patentes''. \url{https://www.gov.br/inpi}

\bibitem{CENTELHA2025} Programa Centelha. (2025). ``Edital Centelha Bahia III''. \url{https://programacentelha.com.br}

\bibitem{FAPESP2025} Fundação de Amparo à Pesquisa do Estado de São Paulo. (2025). ``Programa PIPE - Pesquisa Inovativa em Pequenas Empresas''. \url{https://www.fapesp.br/pipe}

\end{thebibliography}

\end{document}
